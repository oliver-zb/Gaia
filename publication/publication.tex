\documentclass{aastex62}

\usepackage{amsmath}
\newcommand{\vdag}{(v)^\dagger}
\newcommand\aastex{AAS\TeX}
\newcommand\latex{La\TeX}


\submitjournal{RNAAS}

\shorttitle{Local Standard of Rest}
\shortauthors{Zbinden and Saha}

\begin{document}
	
\title{A simple estimate of the Local Standard of Rest using Gaia Radial Velocity Survey}
	
	\correspondingauthor{Oliver Zbinden}
	\email{oliver.zbinden@uzh.ch}
	
	\author{Oliver Zbinden}
	\affil{Physik-Institut, Universit\"at Z\"urich, Winterthurerstrasse 190, 8057 Z\"urich, Switzerland}
	
	
	\author{Prasenjit Saha}
	\affil{Physik-Institut, Universit\"at Z\"urich, Winterthurerstrasse 190, 8057 Z\"urich, Switzerland}
	\affil{Institute for Computational Science, Universit\"at Z\"urich, Winterthurerstrasse 190, 8057 Z\"urich, Switzerland}

\keywords{Galaxy: fundamental parameters, kinematics and dynamics, solar neighborhood}
	
\section{Introduction}\label{intro}
	
It is well known that as stars in the Galactic disk form from gas clouds orbiting in the gravitational field of the Galaxy, younger stars are more likely to be on approximately closed orbits like the gas. On the other hand, older stars follow more eccentric orbits with lower tangential velocities.  The youngest stars in the solar neighborhood should be at the dynamical local standard of rest (LSR), which follows a notional closed orbit around the Galaxy \citep[see e.g.,][]{shu,tayler,keel}.  There are different methods for determination of the LSR, based on the general principle that since hotter stars have a shorter lifetime than colder ones, they are in average younger:  i)~Str\"omgren's method involving finding a relation between mean velocity and velocity dispersion for different stellar populations, and extrapolating to zero velocity dispersion, such as in the now classical work of \cite{delhaye} and recently \cite{ding}, ii)~determining nearly-circular orbits from 6D phase-space data of the stars \citep{francis}, iii)~specifically identifying young OB stars \citep{boby}, iv)~fitting the position and velocity data to a model of the phase-space distribution function \citep{schoenrich}.

With the recent data from the Gaia mission including effective temperatures, another and very simple approach becomes possible: examine the mean velocities in different effective-temperature bins, and if a high-temperature limit is evident, interpret is as the nearly-circular orbits of young stars.
	
\section{Data selection \& Data transformation} 

Our analysis is based on the Gaia~DR2 radial velocity survey.\footnote{\url{http://cdn.gea.esac.esa.int/Gaia/gdr2/gaia_source_with_rv/csv/}}  From this survey we extract stars within $100\rm\,pc$ of our Sun and fractional accuracies of $\sigma_{\mu_{\alpha^*}}/\mu_{\alpha^*},\sigma_{\mu_\delta}/\mu_\delta,\sigma_{\bar\omega}/\bar\omega<0.1$ and $\sigma_{v_{\rm rad}}\leq 5\rm\,km\,s^{-1}$.

To obtain the velocities in cartesian coordinates $\mu_{\alpha^*},\mu_{\delta},v_{\rm rad}$ are transformed using
%	
	\begin{equation}
	\vec{v}_{\rm Equat} = 
	\left(
	\begin{array}{c}
	v_x\\
	v_y\\
	v_z
	\end{array}
	\right) =
	\left(
	\begin{array}{c}
	rv \cos{(\alpha)} \cos{(\delta)} - d k \mu_{\alpha*} \cos{(\alpha)} \sin{(\delta)} - d k \mu_{\delta} \sin{(\alpha)}\\
	rv \sin{(\alpha)} \cos{(\delta)} - d k  \mu_{\delta} \sin{(\alpha)} \sin{(\delta)} + d k \mu_{\alpha*} \cos{(\alpha)}\\
	rv \sin{(\delta)} + d k \mu_{\delta} \cos{(\delta)}
	\end{array}
	\right),	
	\end{equation}
%	
where $d=1/\omega$ is the distance in au, and $k$ is introduced to transform $\rm au\,yr^{-1}$ to $\rm km\,s^{-1}$. These velocities in Equatorial coordinates are transformed to Galactic coordinates $(U,V,W)$ by multiplying with a rotation matrix
\begin{equation}
	\mathbf{\textit{A}}{'}_{G}=\mathbf{\textit{R}}_{z}(-l_{\Omega}) \, \mathbf{\textit{R}}_{x}(90^\circ-\delta_G) \, \mathbf{\textit{R}}_z(\alpha_G+90^\circ)
	\end{equation}
as given in equations (3.60) \& (3.61) of the documentation.\footnote{\url{https://gea.esac.esa.int/archive/documentation/GDR2/}}

% To prevent bad page break
\vskip0pt plus 40pt
\penalty-2000
\vskip0pt plus -40pt
	
\section{Local kinematics and the LSR}

%	Stars similar to our Sun are confined to a thin disk, orbiting the center of the Galaxy \citep{keel}. Because of gravitational potential of all objects in the Milky Way, stars cross the disk in an oscillating motion as well as they do in radial direction. Since all stars are assumed to follow such motions, this should result in a statistically steady state \citep{tayler}.
The stars from the dataset were binned by temperature in steps of $200\rm\,K$, with bins having less than 300 stars neglected. Then, the mean values for each component of $\left( U, V, W\right)$ were calculated for each bin. The temperature-dependent behavior can be found in Figure~\ref{fig:lsrmean}.

	\begin{figure}[ht!]
        \epsscale{1.25}
	\plotone{plots.eps}
	\caption{Heliocentric velocities $U$ (radial), $V$ (azimuthal), $W$ (vertical) for stars with 100~pc of the Sun, as a function of effective temperature.\label{fig:lsrmean}}
	\end{figure}

We see that the radial and vertical velocities have no evident trend with effective temperature.  The azimuthal velocity also has no trend for stars cooler than the Sun, but for hotter stars the mean velocity drops, before levelling off again for $T>6500\rm\,K$.  As described in Section~\ref{intro}, we look at the velocities of high temperature stars. The highest temperature bin with more than 300 stars is $6800\rm\,K$ and has
$$ \left( U, V, W\right)_\odot=\left( 11.9\pm1.2, 11.9\pm0.7, 6.4\pm0.5 \right) \rm\,km\;s^{-1}.$$
This is our estimate of the LSR velocity with respect to the Sun.  Using median rather than mean values within temperature bins gives very similar results. As also shown in Figure~\ref{fig:lsrmean}, this result is in general agreement with recent published values, and in particular is completely consistent with \cite{schoenrich}.

The effective temperature is only an rough criterion for identifying
young stars, since old evolved stars can also go through stages of
high effective temperature.  With that caveat, the temperature-velocity
plot and the estimate of the LSR velocity is interesting.

\bibliographystyle{aasjournal}
\bibliography{bibli.bib}

\end{document}

