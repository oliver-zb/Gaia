\documentclass{aastex62}

\usepackage{amsmath}
\newcommand{\vdag}{(v)^\dagger}
\newcommand\aastex{AAS\TeX}
\newcommand\latex{La\TeX}


\submitjournal{RNAAS}

\shorttitle{Local Kinematics and the LSR}
\shortauthors{Zbinden et al.}

\begin{document}
	
	\title{Local Kinematics and the Local Standard of Rest with \textit{Gaia} Radial Velocity Survey}
	
	\correspondingauthor{Oliver Zbinden \& Prasenjit Saha}
	\affil{Physik Institut, Universit\"at Z\"urich, Winterthurerstrasse 190, 8057 Z\"urich, Switzerland}
	\email{oliver.zbinden@uzh.ch, psaha@physik.uzh.ch}
	
	\author{Oliver Zbinden}
	\affil{Physik Institut, Universit\"at Z\"urich, Winterthurerstrasse 190, 8057 Z\"urich, Switzerland}
	
	
	\author{Prasenjit Saha}
	\affil{Physik Institut, Universit\"at Z\"urich, Winterthurerstrasse 190, 8057 Z\"urich, Switzerland}
	\affil{Institute for Computational Science, Universit\"at Z\"urich, Winterthurerstrasse 190, 8057 Z\"urich, Switzerland}
%	\affil{Institute for Computational Science, Universit\"at Z\"urich, Winterthurerstrasse 190, 8057 Z\"urich, Switzerland}

	\keywords{Galaxy: fundamental parameters, kinematics and dynamics, solar neighborhood}
	
	\section{Introduction}\label{intro}
	
	It is known that stars in the galactic disk form from gas clouds orbiting the galactic center. Thus, younger stars are more likely to be on approximately closed orbits like the gas. On the other hand, older stars follow more eccentric orbits with higher tangential velocities. Since hotter stars have a shorter lifetime than colder ones, we assume that they are in average younger. The local standard of rest (LSR) follows the mean motion of the disk material in the solar neighborhood \citep{shu}, and there are different methods for its determination. i) Str\"omgrens method involving finding a relation between mean velocity and velocity dispersion for different stellar
	populations, and extrapolating to zero velocity dispersion (e.g. \cite{ding}), ii) determine nearly-circular orbits from 6D phase-space data of the stars (e.g. \cite{francis}), iii) specifically observing young OB stars (e.g. \cite{boby}), iv) fitting the data to a model of the phase-space distribution function.
	\\
	However, since the amount of data provided by the \textit{Gaia} mission \citep{gaiamission} is significantly larger than previous data sets, we used a much simpler method. Since we are interested in nearly-circular orbits of young stars, we look at the velocities of high temperature stars.
	
	\section{Data selection \& Data transformation} 

	Our analysis is based on data from the \textit{Gaia} DR2 radial velocity survey\footnote{\url{http://cdn.gea.esac.esa.int/Gaia/gdr2/gaia_source_with_rv/csv/}}, closer than $100\rm pc$ away from our Sun. Furthermore, all stars with either $\frac{\sigma_{\alpha^*}}{\alpha^*}$,$\frac{\sigma_{\delta}}{\delta}$ or $\frac{\sigma_{\bar{\omega}}}{\bar{\omega}}$ larger than $0.1$, and $\sigma_{v_{rad}}\leq$ $|5|$ $\rm km~s^{-1}$ are neglected.
	\\
	To obtain the velocities $\mu_{\alpha^*}$, $\mu_{\delta}$ and $v_{rad}$ are transformed to Cartesian coordinates using
	
	\begin{equation}
	\vec{v}_{Equat} = 
	\left(
	\begin{array}{c}
	v_x\\
	v_y\\
	v_z
	\end{array}
	\right) =
	\left(
	\begin{array}{c}
	rv \cos{(\alpha)} \cos{(\delta)} - d k \mu_{\alpha*} \cos{(\alpha)} \sin{(\delta)} - d k \mu_{\delta} \sin{(\alpha)}\\
	rv \sin{(\alpha)} \cos{(\delta)} - d k  \mu_{\delta} \sin{(\alpha)} \sin{(\delta)} + d k \mu_{\alpha*} \cos{(\alpha)}\\
	rv \sin{(\delta)} + d k \mu_{\delta} \cos{(\delta)}
	\end{array}
	\right),	
	\end{equation}
	
	\noindent
	in $\rm km~s^{-1}$ where $d$ is the distance, and $k$ is introduced to transform $\rm au~yr^{-1}$ to $\rm km~s^{-1}$. These velocities in Equatorial coordinates are transformed to Galactic coordinates by multiplying with a rotation matrix
	
	\begin{equation}
	\mathbf{\textit{A}}{'}_{G}=\mathbf{\textit{R}}_{z}(-l_{\Omega})\mathbf{\textit{R}}_{x}(90^\circ-\delta_G)\mathbf{\textit{R}}_z(\alpha_G+90^\circ).
	\end{equation}
	
	given in equations 3.60 \& 3.61 of the \textit{Gaia} DR2 documentation \citep{documentation}.
	
	\section{Local kinematics}

	Stars similar to our Sun are confined to a thin disk, orbiting the center of the Galaxy \citep{keel}. Because of gravitational potential of all objects in the Milky Way, stars cross the disk in an oscillating motion as well as they do in radial direction. Since all stars are assumed to follow such motions, this should result in a statistically steady state \citep{tayler}. The stars from the dataset were grouped by temperature in steps of $200K$, where groups with less than $300$ objects were neglected. Then, the mean values for each component of $\left( U, V, W\right)$ were calculated for each group. The temperature-dependent behavior can be found in figure \ref{fig:lsrmean}.

	\begin{figure}[ht!]
	\plotone{plots.eps}
	%		\caption{}
	\label{fig:lsrmean}
	\end{figure}

	\section{Local Standard of Rest} \label{lsr}

	As described in section \ref{intro}, we look at the velocities of high temperature stars. The highest temperature with sufficient data is $6800\rm K$ with $\left( U, V, W\right)_\odot=\left( 11.92\pm1.17, 11.87\pm0.72, 6.44\pm0.46 \right)\rm km~s^{-1}$.
	\\
	Compared to other results, e.g. \citep{schoenrich}, \citep{boby}, \citep{francis} or \citep{ding}, one can see that our values are within uncertainties with \textit{Sch\"onrich et al.} which probably enjoys the highest acceptance. However, as one can see there is not really a consensus for these velocities and it is not clear how the different results should be interpreted.

	\section{Conclusion}

	It was shown that the motion of nearby stars is almost independent of temperature for stars with T$\leq 5800\rm K$. At this temperature, $V_\odot$ drops significantly. Since there is not enough data on the high-temperature area, it is not possible to make predictions how velocities change there.
	\\
	As shown in section \ref{lsr}, recent results of the LSR are not consistent. Therefore, it is not clear if we really have found a simple but robust way to determine the LSR. It would be interesting to test our method again, once there is more data accessible for stars with temperatures above $6800\rm K$.
	
	\begin{thebibliography}{}
		
		\bibitem[Gaia Collaboration et al. (2016)]{gaiamission} Gaia Collaboration et al. 2016, Description of the Gaia mission (spacecraft, instruments, survey and measurement principles, and operations)
		\bibitem[Gaia DR2 Documentation (2018)]{documentation} European Space Agency \& Gaia Data Processing and Analysis Consortium, 2018, Documentation release 1.0
		\bibitem[Keel (2007)]{keel}W.C. Keel, 2007, The Road to Galaxy Formation, Springer, Second edition
		\bibitem[Tayler (1993)]{tayler} R. J. Tayler, Galaxies: structure and evolution, Cambridge University Press
		\bibitem[Shu (1982)]{shu} F. H. Shu, 1982, The Physical Universe - An Introduction to Astronomy, University Science Books
		\bibitem[Sch\"onrich et al. (2009)]{schoenrich} R. Sch\"{o}nrich, J, Binney \& W. Dehnen, 2009, Local Kinematics and the Local Standard of Rest, arXiv:0912.3693v1 [astro-ph.GA]
		\bibitem[Bobylev \& Bajkova (2018)]{boby} V. V. Bobylev \& A. T. Bajkova, 2018, Kinematics of the Galaxy from OB Stars with Data from the Gaia DR2 Catalogue, arXiv:1809.10512v1 arXiv:1809.10512v1
		\bibitem[Francis \& Anderson (2018)]{francis} C. Francis \& E. Anderson, 2018, The local standard of rest and the well in the velocity distribution, arXiv:1311.2069v2 \textcolor{red}{in paper: 2018, on arxiv since 2013!?}
		\bibitem[Ding et al. (2018)]{ding} P.-J. Ding, Z. Zhu \& J.-C. Liu, 2018, Local standard of rest based on Gaia DR2 catalog, Research in Astronomy and Astrophysics
		
	\end{thebibliography}
	
\end{document}